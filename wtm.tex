\documentclass[a5paper,8pt]{article}
\usepackage[utf8]{inputenc}
\usepackage{lmodern}
\usepackage[MeX]{polski}
\usepackage{amsmath}
\usepackage{amsthm}
\usepackage{amsfonts}
\usepackage{cases}
\usepackage{geometry}
\usepackage{enumitem}
\usepackage{mathtools}
\usepackage{amssymb}

\newgeometry{tmargin=2cm, bmargin=2cm, lmargin=1.5cm, rmargin=1.5cm}
\setlength{\parindent}{0cm}

% \newtheoremstyle{mythmstyle}%
%     {}%
%     {}%
%     {\it}%
%     {}%
%     {\bf}%
%     {}%
%     { }%
%     {\thmname{#1}\thmnumber{ #2}%
%     \thmnote{: #3\addcontentsline{toc}{subsubsection}{{\it#1}: #3}}.\hfill \\ }
% \theoremstyle{mythmstyle}


\newtheoremstyle{mythmstyle}%
    {}%
    {}%
    {\it}%
    {}%
    {\bf}%
    {}%
    { }%
    {\thmname{#1}. \normalfont #3 \hfill \\ }
\theoremstyle{mythmstyle}

% \newtheorem{example}{Przykład}[section]
\newtheorem{lemat}{Lemat}[section]
\newtheorem{definition}{Definicja}[section]
\newtheorem{theorem}{Twierdzenie}[section]
\newtheorem*{conclusion}{Wniosek}
\newtheorem*{example}{Przykład}

\newcommand\ass{\mathrel{\overset{\makebox[0pt]{\mbox{\normalfont\tiny\sffamily zał}}}{=}}}

\newcommand\tab[1][1cm]{\hspace*{#1}}

\title{Wstęp do teorii mnogości}
\author{Stanisław Migórski}
\date{}
\frenchspacing

\begin{document}
    \maketitle
    \tableofcontents
    \pagebreak

    \section{Program} % (fold)
    \label{sec:program}
        \begin{enumerate}
            \item Dowody i elementy logiki.
            \item Zbiory i działania na nich.
            \item Relacje równoważności.
            \item Funkcje.
            \item Własności funkcji.
            \item Zbiory równoliczne i nierównoliczne.
            \item Relacje porządku.
            \item Konstukcje liczbowe.
            \item Lemat Kuratowskiego-Zorna.
        \end{enumerate}
    % section program (end)

    \section{Literatura} % (fold)
    \label{sec:literatura}
        \begin{enumerate}
            \item K. Kustowski, A. Mostowski, \textit{Teoria mnogości}, PWN, 1994
            \item H. Rosiowa, \textit{Wstęp do matematyki}, PWN, 2004
            \item W. Marek, J. Onyszkiewicz, \textit{Elementy logiki i teorii mnogości w zadaniach}, PWN, 1996
        \end{enumerate}
    % section literatura (end)

    \section{Zasady oceniania} % (fold)
    \label{sec:zasady_oceniania}

        \begin{equation*}
          \textbf{WTM:  }\begin{cases}
            & \text{ćwiczenia 30h, \textbf{2} obecności bez usprawiedliwienia}.\\
            & \text{wykład 30h}.
          \end{cases}
        \end{equation*}

        \textbf{Ocenia końcowa: } \textbf{20\%} \underline{oceny} z ćwiczeń + \textbf{80\%} \underline{oceny} z egzaminu (I, II termin).\\

        \textbf{Pegaz: } zestawy zadań: A, B - obowiązkowe. Dowody do oceny "DDO".

    % section zasady_oceniania (end)

    \section{Spójniki logiczne} % (fold)
    \label{sec:spójniki_logiczne}

        \subsection{Standardowe} % (fold)
        \label{sub:standardowe}
            $\neg$ (negacja), $\wedge$ (koniunkcja), $\vee$ (alternatywa), $\implies$ (implikacja), $\iff$ (równoważność)
        % subsection standardowe (end)

        \subsection{Inne spójniki} % (fold)
        \label{sub:inne_spójniki}
            \begin{enumerate}
                \item \textbf{Alternatywa rozłączna} $\alpha$ i $\beta$, oznaczamy $\alpha \oplus \beta$.\\
                Czytamy "... albo ..." lub "albo ..., albo ...".\\

                \begin{center}
                    \begin{tabular}{|c|c|c|}
                      \hline 
                      $\alpha$ & $\beta$ & $\alpha \oplus \beta$\\
                      \hline
                      0 & 0 & 0\\
                      \hline
                      0 & 1 & 1\\
                      \hline
                      1 & 0 & 1\\
                      \hline
                      1 & 1 & 0\\
                      \hline
                    \end{tabular}
                \end{center}

                \begin{equation*}
                    (\alpha \oplus \beta) \iff (\neg \alpha \iff \beta) \iff (\alpha \iff \neg \beta) \iff (\neg (\alpha \iff \beta))
                \end{equation*}


                \item \textbf{Dyzjunkcja} \textit{(kreska Sheffera)} $\alpha$ i $\beta$, oznaczamy $\alpha \mid \beta$.\\
                Czytamy "albo nie ..., albo nie .."\\

                \begin{center}
                    \begin{tabular}{|c|c|c|}
                      \hline 
                      $\alpha$ & $\beta$ & $\alpha \mid \beta$\\
                      \hline
                      0 & 0 & 1\\
                      \hline
                      0 & 1 & 1\\
                      \hline
                      1 & 0 & 1\\
                      \hline
                      1 & 1 & 0\\
                      \hline
                    \end{tabular}
                \end{center}

                \begin{equation*}
                    (\alpha \mid \beta) \iff (\neg (\alpha \wedge \beta))
                \end{equation*}

                \pagebreak
                \item \textbf{Binegacja} \textit{(strzałka Pierce'a, funktor Łukasiewicza)} $\alpha$ i $\beta$, oznaczamy $\alpha \downarrow \beta$.\\
                Czytamy "ani ..., ani .."\\

                \begin{center}
                    \begin{tabular}{|c|c|c|}
                      \hline 
                      $\alpha$ & $\beta$ & $\alpha \downarrow \beta$\\
                      \hline
                      0 & 0 & 1\\
                      \hline
                      0 & 1 & 0\\
                      \hline
                      1 & 0 & 0\\
                      \hline
                      1 & 1 & 0\\
                      \hline
                    \end{tabular}
                \end{center}

                \begin{equation*}
                    (\alpha \downarrow \beta) \iff (\neg \alpha \wedge \neg \beta)
                \end{equation*}

            \end{enumerate}
        % subsection inne_spójniki (end)
            
        \subsection{Związki z OAK} % (fold)
        \label{sub:związki_z_oak}
            Spójniki logiczne mają przyporządkowane \underline{bramki logiczne}.

            \begin{itemize}
                \item \textbf{NOT: } $ \alpha \longmapsto \neg \alpha $ (negacja)
                \item \textbf{AND: } $ \alpha, \beta \longmapsto \alpha \wedge \beta $ (koniunkcja)
                \item \textbf{NAND: } $ \alpha, \beta \longmapsto \neg (\alpha \wedge \beta) $ (dyzjunkcja)
                \item \textbf{OR: } $ \alpha, \beta \longmapsto \alpha \vee \beta $ (alternatywa)
                \item \textbf{NOR: } $ \alpha, \beta \longmapsto \neg (\alpha \vee \beta) \iff (\neg \alpha \wedge \neg \beta) \iff (\alpha \downarrow \beta) $ (binegacja)
                \item \textbf{XOR: } $ \alpha, \beta \longmapsto \alpha \oplus \beta $ (alternatywa rozłączna)
                \item \textbf{XNOR: } $ \alpha, \beta \longmapsto \neg (\alpha \oplus \beta) $ (negacja alternatywy rozłącznej)
            \end{itemize}
        % subsection związki_z_oak (end)

    % section spójniki_logiczne (end)

    \pagebreak
    \section{Rachunek funkcyjny} % (fold)
    \label{sec:rachunek_funkcyjny}

        \subsection{Funkcja zdaniowa} % (fold)
        \label{sub:funkcja_zdaniowa}
        
        Niech $ x_1, \ldots, x_n $ będą zbiorami.

        \begin{definition}[Funkcja zdaniowa]
            Funkcją (formą) zdaniową $ n $ zmiennych nazywamy wyrażenie (formułę) $ \varphi (x_1, \ldots, x_n) $, w którym występuje $ n $ zmiennych $ x_1, \ldots, x_n $, które zmienia się w zdanie logiczne, gdy za zmienne $ x_1, \ldots, x_n $ podstawimy nazwę dowolnego elementu ze zbiorów $ X_1, \ldots, X_n $.
        \end{definition}

        \begin{definition}[Dziedzina funkcji zdaniowej]
            \textbf{Dziedziną} (zakresem zmienności) funkcji zdaniowej $ \varphi (x_1, \ldots, x_n) $ nazywamy iloczyn kartezjański $ x_1 \times \ldots \times x_n $ i zapisujemy $ Z(\varphi) = x_1 \times \ldots \times x_n $.
        \end{definition}

        \begin{definition}
            Mówimy, że n-tka uporządkowona \\
            $ (a_1, \ldots, a_n) \in X_1 \times \ldots \times X_n $ spełnia funkcję zdaniową $ \varphi (x_1, \ldots, x_n) $, jeżeli zdanie $ \varphi (a_1, \ldots, a_n) $ jest prawdziwe.
        \end{definition}

        \begin{definition}[Zbiór spełniania funkcji zdaniowej]
            \textbf{Zbiór spełniania} funkcji zdaniowej $ \varphi (x_1, \ldots, x_n) $ określamy następująco:

            \begin{equation*}
                S(\varphi) = \{ (a_1, \ldots, a_n) \in X_1 \times \ldots \times X_n : \varphi (a_1, \ldots, a_n) = 1 \}
            \end{equation*}

            Funkcja zdaniowa $ \varphi (x_1, \ldots, x_n) $ jest prawdziwa w zbiorze $ X_1 \times \ldots \times X_n $, jeżeli $ S(\varphi) = X_1 \times \ldots \times X_n $.

        \end{definition}
        
        \begin{theorem}
            Niech $ \varphi(x_1 \times \ldots \times x_n) $, $ \psi(x_1 \times \ldots \times x_n) $, gdzie $ x_i \in X_i $, $ i = 1, \ldots, n $ będą funkcjami zdaniowymi. Wtedy:

            \begin{enumerate}[label=\textbf{\arabic*})]
                \item $ S( \varphi \wedge \psi ) = S( \varphi ) \cap S( \psi ) $
                \item $ S( \varphi \vee \psi ) = S( \varphi ) \cup S( \psi ) $
                \item $ S( \neg \varphi ) = (X_1 \times \ldots \times X_n ) \setminus S( \varphi ) $
                \item $ S( \varphi \implies \psi ) = (( X_1 \times \ldots \times X_n ) \setminus S( \varphi ) ) \cup S( \psi ) $
                \item $ S( \varphi \iff \psi ) = ( S( \varphi ) \cap S( \psi ) ) \cup ( ( ( X_1 \times \ldots \times X_n ) \cap ( ( X_1 \times \ldots \times X_n ) \setminus S( \psi ) ) ) $%(( X_1 \times \ldots \times X_n ) \setminus S( \varphi ) ) \cup S( \psi ) $
            \end{enumerate}

        \end{theorem}

        \begin{definition}
            Funkcje zdaniowe $ \varphi (x_1, \ldots, x_n) $ i $ \psi (x_1, \ldots, x_n) $ nazywamy \textbf{równoważnymi} jeżeli

            \begin{equation*}
                S(\varphi) = S(\psi).
            \end{equation*}

            Zapisujemy:

            \begin{equation*}
                \varphi (x_1, \ldots, x_n) \equiv \psi (x_1, \ldots, x_n)
            \end{equation*}

        \end{definition}

        % subsection funkcja_zdaniowa (end)

        \subsection{Kwantyfikatory} % (fold)
        \label{sub:kwantyfikatory}
            \begin{itemize}
                \item Kwantyfikator ogólny: {\LARGE $ \forall $ }, $ \bigwedge $ (dla każdego)
                \item Kwantyfikator szczególny: {\LARGE $ \exists $ }, $ \bigvee $ (istnieje)\\
                \tab \tab \tab \tab {\LARGE $ \exists ! $ }, (istnieje dokładnie jeden)
            \end{itemize}
        % subsection kwantyfikatory (end)

        \subsection{Prawa rachunku funkcyjnego} % (fold)
        \label{sub:prawa_rachunku_funkcyjnego}

            \begin{theorem}[Prawa de Morgana]
                \[
                    \begin{dcases*}
                    \neg( \exists x \in X : \varphi(x)) \iff \forall x \in X : \neg \varphi(x) \\
                    \neg( \forall x \in X : \varphi(x)) \iff \exists x \in X : \neg \varphi(x)
                    \end{dcases*}
                \]
            \end{theorem}

            \begin{theorem}[Prawo egzemplifikacji]
                \begin{equation*}
                    ( \forall x \in X : \varphi(x) ) \implies ( \exists x \in X : \varphi(x) )
                \end{equation*}
            \end{theorem}

            \begin{theorem}[Prawo przestawiania kwantyfikatorów]

                \begin{equation*}
                    (\forall x \in X, \forall y \in Y : \varphi(x, y)) \iff ( \forall y \in Y, \forall x \in X : \varphi(x, y))
                \end{equation*}

                \begin{equation*}
                    (\exists x \in X, \exists y \in Y : \varphi(x, y)) \iff ( \exists y \in Y, \exists x \in X : \varphi(x, y))
                \end{equation*}

                \begin{equation*}
                    (\exists x \in X, \forall y \in Y : \varphi(x, y)) \implies ( \forall y \in Y, \exists x \in X : \varphi(x, y))
                \end{equation*}

                \begin{center}
                    $ \impliedby $ nie zachodzi!!
                \end{center}

            \end{theorem}

            \begin{theorem}[Prawo włączania i wyłączania kwantyfikatorów]

                \[
                    \begin{dcases*}
                    \forall x \in X : (\varphi(x) \vee \psi) \iff (\forall x \in X : \varphi(x)) \vee \psi \\
                    \exists x \in X : (\varphi(x) \vee \psi) ? (\exists x \in X : \varphi(x)) \vee \psi \\
                    \ldots
                    \end{dcases*}
                \]

            \end{theorem}

            \begin{theorem}[Prawo rodzielności kwantyfikatora ogólnego względem koniunkcji]

                \[
                    \begin{dcases*}
                    \forall x \in X : (\varphi(x) \wedge \psi(x)) \iff (\forall x \in X : \varphi(x)) \wedge (\forall x \in X : \psi(x)) \\
                    \forall x \in X : (\varphi(x) \vee \psi(x)) \impliedby (\forall x \in X : \varphi(x)) \vee (\forall x \in X : \psi(x)) \\
                    \ldots
                    \end{dcases*}
                \]

            \end{theorem}

            \begin{example}[Prawo rozdzielności kwantyfikatora ogólnego względem implikacji]
                \begin{equation*}
                    (\forall x \in X : ( \varphi(x) \implies \psi(x))) \implies ((\forall x \in X : \varphi(x)) \implies (\forall x \in X : \psi(x))
                \end{equation*}

                \begin{center}
                    $ \impliedby $ nie zachodzi
                \end{center}

                Niech:
                \begin{equation*}
                    \begin{aligned}
                        & \varphi(x) = \{ x \in \mathbb{R} : x < 0 \} \\
                        & \psi(x) = \{ x \in \mathbb{R} : x + 1 > 0 \} \\
                        & X = \mathbb{R}
                    \end{aligned}
                \end{equation*}

                Wtedy:
                \begin{equation*}
                    \underbrace{\underbrace{(\forall x \in \mathbb{R} : (x < 0 \implies x + 1 > 0 )}_{\text{fałsz}})
                    \impliedby \underbrace{(\underbrace{(\forall x \in \mathbb{R} : x < 0)}_{\text{fałsz}} \implies \underbrace{(\forall x \in \mathbb{R} : x + 1 > 0)}_{\text{fałsz}})}_{\text{prawda}}}_{\text{fałsz}}
                \end{equation*}

            \end{example}

            \pagebreak
            \begin{example}[Prawo rodzielności kwantyfikatora szczególnego względem koniunkcji]
                \begin{equation*}
                    (\exists x \in X : ( \varphi(x) \wedge \psi(x))) \implies ((\exists x \in X : \varphi(x)) \wedge (\exists x \in X : \psi(x))
                \end{equation*}

                \begin{center}
                    $ \impliedby $ nie zachodzi
                \end{center}

                Niech:
                \begin{equation*}
                    \begin{aligned}
                        & \varphi(x) = \{ 2 \mid x : x \in \mathbb{N} \} \text{parzyste}\\
                        & \psi(x) = \{ \neg 2 \mid x : x \in \mathbb{N} \} \text{nieparzyste} \\
                        & X = \mathbb{N}
                    \end{aligned}
                \end{equation*}

                Wtedy:
                \begin{equation*}
                    \underbrace{\underbrace{(\exists x \in \mathbb{N} : (2\mid x \wedge \neg 2\mid x )}_{\text{fałsz}})
                    \impliedby \underbrace{(\underbrace{(\exists x \in \mathbb{N} : 2 \mid x)}_{\text{prawda}} \wedge \underbrace{(\exists x \in \mathbb{N} : \neg 2 \mid x)}_{\text{prawda}})}_{\text{prawda}}}_{\text{fałsz}}
                \end{equation*}

            \end{example}

            {\large{\textbf{Uwaga:}}}
            \begin{enumerate}
                \item Dla formy zdaniowej jednej zmiennej zachodzi:
                \begin{equation*}
                    (\forall x : P(x) \implies (\exists x \in X : P(x)))
                \end{equation*}

                \item Dla formy zdaniowej dwóch zmiennych zachodzi:

                \begin{equation*}
                    \begin{aligned}
                        &( \forall x, \forall y : P(x, y) ) \iff &( \forall y, \forall x : P(x,y)) \\
                        & \Downarrow & \Downarrow \\
                        &( \exists x, \forall y : P(x, y) ) & ( \exists y, \forall x : P(x,y)) \\
                        & \Downarrow & \Downarrow \\
                        &( \forall y, \exists x : P(x, y) ) & ( \forall x, \exists y : P(x,y)) \\
                        & \Downarrow & \Downarrow \\
                        &( \exists y, \exists x : P(x, y) ) & ( \exists x, \exists y : P(x,y))
                    \end{aligned}
                \end{equation*}

            \end{enumerate}

        % subsection prawa_rachunku_funkcyjnego (end)

    % section rachunek_funkcyjny (end)

    \pagebreak
    \section{Zbiory i działania na zbiorach} % (fold)
    \label{sec:zbiory_i_dzialania_na_zbiorach}
    % section zbiory_i_dzialania_na_zbiorach (end)

    \subsection{Pojęcia} % (fold)
    \label{sub:pojęcia}
    
    % subsection pojęcia (end)
        \begin{itemize}
            \item Pojęcie zbioru jest pojęciem \underline{pierwotnym}.
                \begin{align*}
                    & \mathbb{N}, \mathbb{Z}, \mathbb{Q}, \mathbb{R}, \mathbb{C} - \text{zbiory liczbowe} \\
                    & \mathbb{N} = \{0, 1, 2, \ldots\} \text{\tab \textit{(Pezno)}} \\
                    & \mathbb{N}_{1} = \{1, 2, \ldots\}
                \end{align*}

            \item Pojęcie "należenia do zbioru"
                \begin{equation*}
                    1 \in \mathbb{N}, (\neg (\sqrt{2} \in \mathbb{N} )) \iff (\sqrt{2} \notin \mathbb{N})
                \end{equation*}

            \item Zasada równości zbiorów
                \begin{align*}
                    & A = B, tzn. \forall x : \underbrace{(x \in A \implies x \in B)}_{A \subset B} \wedge \underbrace{(x \in B \implies x \in A)}_{B \subset A} \\
                    & A \subset B, wtw. \forall x : (x \in A \implies x \in B)
                \end{align*}

                To znaczy, aby udowodnić, że $ \mathbf{A = B} $ dowodzimy
                \begin{equation*}
                  \forall x \begin{cases}
                    & x \in A \implies x \in B \\
                    & x \in B \implies x \in A
                  \end{cases} \\
                \end{equation*}
                \begin{equation*}
                    A \subset B \wedge B \subset A
                \end{equation*}

            \item Oznaczamy
                \begin{align*}
                    & \{a, b\} \tab & \text{para uporządkowana} \\
                    & \{a\} & \text{singleton} \\
                    & \{a, a\} = \{a\}  \\
                \end{align*}

            \item Definiowanie zbiorów
                \begin{enumerate}
                    \item Wypisujemy elementy zbioru
                    \item $ \mathbf{A = \{ x \in X : W(x) \} } $ \\
                    $ \text{Par} = \{x \in \mathbb{N} : \underbrace{2 \mid x}_{W(x)} \}, \tab \{ \underbrace{ \{1, 2\} }_{\text{jeden element}}\} \neq \{1, 2\} $
                \end{enumerate}

        \end{itemize}
            % \item \textbf{Zbiór potęgowy}
        % \noindent\rule{\textwidth}{1pt}

        \begin{definition}[Zbiór potęgowy]
            Zbiorem potęgowym zbioru A nazywamy zbiór złożony ze wszystkich pozbiorów zbioru A, tzn. $ \mathcal{P}(A) = \{X : X \subset A \} $
        \end{definition}

        Zbiór potęgowy $ \mathcal{P}(A) $ zbioru $ A $.
        \begin{align*}
            & \{ \varnothing, A \} \subset \mathcal{P}(A) \longleftarrow \\
            & \mathcal{P}(\varnothing) = \{\varnothing\} \\
            & \mathcal{P}(\{1\}) = \{\varnothing, \{1\}\} \\
            & \mathcal{P}(\{1, 2\}) = \{\varnothing, \{1\}, \{2\}, \{1, 2\} \} \\
            & \ldots
        \end{align*}

        \subsection{Działania na zbiorach} % (fold)
        \label{sub:dzialania_na_zbiorach}

        \begin{definition}[Suma zbiorów]
            Sumą zbiorów $A$ i $B$ nazywamy zbiór złożony z tych i tylko z tych elementów, które należą do co najmniej jednego z tych zbiorów.

            \begin{equation*}
                A \cup B = \{ x : x \in A \vee x \in B \}
            \end{equation*}

        \end{definition}

        \begin{theorem}
            Jeżeli $A$ i $B$ są zbiorami, to wtedy:

            \begin{enumerate}
                \item $ A \subset A \cup B, \tab B \subset A \cup B $

                \item $ ( A \subset C \wedge B \subset C ) \implies A \cup B \subset C $
            \end{enumerate}
        \end{theorem}

        Niech $ \mathcal{A} $ będzie rodziną zbiorów (tzn. zbiorami, którego elemantami są zbiory).

        \begin{definition}[Suma rodziny]
            Suma rodziny $ \mathcal{A} $ jest zbiorem złożonym z tych i tylko tych elementów, które należą do co najmniej jednego spośród zbiorów należących do rodziny $ \mathcal{A} $, tzn. $ x \in \bigcup \mathcal{A} $ wtw. $ \exists A \in \mathcal{A} : x \in \mathcal{A} $
        \end{definition}

        Jest uogólnienie pojęcia sumy zbiorów.

        \begin{equation*}
            \bigcup \{ A, B \} = A \cup B, \tab \bigcup \{ A, B, C \} = A \cup B \cup C
        \end{equation*}

        \pagebreak
        \begin{theorem}
            \textbf{Jeżeli} $ \mathcal{A} $ jest rodziną zbiorów, \underline{to} wtedy

            \begin{enumerate}
                \item $ \forall A : ( A \in \mathcal{A} \implies A \subset \bigcup \mathcal{A} ) $

                \item \textbf{Jeżeli} C jest zbiorem o następującej własności:
                    \begin{equation*}
                        \forall A : ( A \in \mathcal{A} \implies A \subset C )
                    \end{equation*}

                    \textbf{To} \tab $ \bigcup \mathcal{A} \subset C $
            \end{enumerate}
        \end{theorem}

        \begin{definition}[Iloczyn zbiorów]
            Iloczynem zbiorów (przecięciem, częścią wspólną) A i B nazywamy zbiór złożony z tych i tylko z tych elementów, które należą jednocześnie do A i B.

            \begin{equation*}
                A \cap B = \{ x : x \in A \wedge x \in B \}
            \end{equation*}

        \end{definition}

        \begin{theorem}
            Jeżeli A i B są zbiorami, to

            \begin{enumerate}
                \item $ A \cap B \subset A, A \cap B \subset B $

                \item $ ( C \subset A \wedge C \subset B ) \implies C \subset A \cap B $
            \end{enumerate}
            
            Jeżeli $ A \cap B = \varnothing $, to A i B nazywamy rozłącznymi.                 
        \end{theorem}


        Niech $ \mathcal{A} $ będzie niepustą rodziną zbiorów.

        \begin{definition}[Iloczyn rodziny]
            Iloczynem rodziny $ \mathcal{A} $ nazywamy zbiór złożony z tych i tylko z tych
            elementów, które należą do każdego spośród zbiórów należących do rodziny $ \mathcal{A} $,
            tzn.

            \begin{equation*}
                x \in \bigcap \mathcal{A} \iff \forall A \in \mathcal{A} : x \in A
            \end{equation*}
        \end{definition}

        Jest uogólnienie pojęcia iloczynu zbiorów.
        \begin{equation*}
            \bigcap \{ A, B \} = A \cap B, \tab \bigcap \{ A, B, C \} = A \cap B \cap C
        \end{equation*}

        \pagebreak
        
        \begin{theorem}
            \textbf{Jeżeli} $ \mathcal{A} $ jest niepustą rodziną zbiorów, \underline{to} wtedy

            \begin{enumerate}
                \item $ \forall A : ( A \in \mathcal{A} \implies \bigcap \mathcal{A} \subset A ) $

                \item \textbf{Jeżeli} C jest zbiorem o następującej własności:
                    \begin{equation*}
                        \forall A : ( A \in \mathcal{A} \implies C \subset A )
                    \end{equation*}

                    \textbf{To} \tab $ C \subset \bigcap \mathcal{A} $
            \end{enumerate}
        \end{theorem}


        \begin{definition}[Różnica zbiorów]
            Różnicą zbiorów A i B nazywamy zbiór złożony z tych i tylko z tych
            elementów, które należą do A i nie należą do B, tzn.

            \begin{equation*}
                A \setminus B = \{ x: x \in A \wedge \neg x \in B \} = \{ x: x \in A \wedge x \notin B \}
            \end{equation*}
        \end{definition}

        \begin{theorem}
            Jeżeli $ A \wedge B $ są zbiorami, to

            \begin{enumerate}
                \item $ A \setminus B \subset A, (A \setminus B) \cap B = \varnothing $

                \item $ C \subset A, C \cap B \neq \varnothing ) \implies C \subset A \setminus B $
            \end{enumerate}
        \end{theorem}

        \begin{definition}[Dopełnienie zbioru]
            Dopełnieniem (uzupełnieniem) zbioru A w zbiorze S nazywamy zbiór $ S \setminus A $, piszemy

            \begin{equation*}
                A' = S \setminus A = \setminus A
            \end{equation*}
        \end{definition}

        \begin{definition}[Różnica symetryczna]
            Różnicą symetryczną zbiorów A i B nazywamy zbiór zdefiniowany:
            \begin{align*}
                & A \bigtriangleup B = A \div B = ( A \setminus B ) \cup ( B \setminus A ) \\
                & x \in A \div B \iff x \in A \setminus B \vee x \in B \setminus A \\
                & A \div B = \varnothing \iff A = B
            \end{align*}
        \end{definition}

        % subsection działania_na_zbiorach (end)

        \pagebreak
        \subsection{Zbiory skończone} % (fold)
        \label{sub:zbiory_skończone}
            
        \begin{definition}[Zbiór skończony]
            Zbiór A nazywamy zbiorem skończonym jeżeli A zawiera m różnych elementów, gdzie $ m \in \{ 0, 1, 2, \ldots \} $.
        \end{definition}
        \begin{align*}
            \text{\textbf{Oznaczamy: }} & n(A) \text{ - liczba elementów skończonego zbioru} \\
            & A = |A| = \bar{\bar{A}} = \#A
        \end{align*}

        \textbf{Własności.} \\
        Jeżeli A, B, C są zbiorami skończonymi, to

        \begin{enumerate}
            \item $ A \cup B $, $ A \cap B $ są zbiorami skończonymi, \\
                  $ n(A \cup B) = n(A) + n(B) - n(A \cap B) $.\\\\
                  Jeżeli A i B są rozłączne, to $ n(A \cup B) = n(A) + n(B) $.
            \item Zbiór A jest podzbiorem skończonego zbioru E. \\
                  Wtedy $ n(A') = n(E) - n(A) $.
            \item $ n(A \setminus B) = n(A) - n(A \cap B) $.
            \item $ n(A \cup B \cup C) = n(A) + n(B) + n(C) - n(A \cap B) - n(A \cap C) $ \\ \tab\tab $ - n(B \cap C) + n(A \cap B \cap C) $.
            \item $ n(\mathcal{P}(A)) = 2^{n(A)} $.

        \end{enumerate}
        % subsection zbiory_skończone (end)

        \pagebreak
        \section{Para uporządkowana} % (fold)
        \label{sec:para_uporządkowana}

        \subsection{Para uporządkowana} % (fold)
        \label{sub:para_uporządkowana}
        

            \begin{definition}[Para uporządkowana (Kuratowski)]
                Parą uporządkowaną elementów a i b nazywamy $ (a, b) = \{ \{a\}, \{a, b\} \} $.
            \end{definition}

            \begin{theorem}
                Dla każdego a, b, c, d zachodzi
                \begin{equation*}
                    \{ \{a\}, \{a, b\} \} = \{ \{c\}, \{c, d\} \} \iff a = c \wedge b = d
                \end{equation*}
                \begin{equation*}
                    (a, b) = (c, d)
                \end{equation*}
            \end{theorem}

            \textbf{Dowód.}
            \begin{itemize}
                \item $ \impliedby $ oczywiste
                \item $ \implies $
                    \begin{enumerate}
                        \item Jeżeli $ a = b $, to $ \{ \{a\}, \{a, b\} \} = \{ \{a\} \}\tab (\{a, a\} = \{a\} )$ \\
                              oraz $ \{c, d\} \in \{ \{a\} \} $. Wtedy $ c = d = a $. Zatem $ a = b = c =d $.

                        \item Jeżeli $ a \neq b $, to $ \{c\} \in \{ \{a\}, \{a, b\} \} $. \\
                              Ponieważ $ a \neq b $, to $ \{c\} \notin \{a, b\} $. \\
                              Wtedy $ \{c\} =  \{a\} $, tzn. $ \underline{a = c} $. \\
                              Z drugiej strony, $ \{a, b\} \in \{ \{c\}, \{c, d\} \} $.\\
                              Ponieważ $ a \neq b $, to $ \{a, b\} = \{c, d\} $ oraz $ \underline{b = d} $.
                    \end{enumerate}
            \end{itemize}

        % subsection para_uporządkowana (end)

        \subsection{Iloczyn kartezjański} % (fold)
        \label{sub:iloczyn_kartezjański}
            
            \begin{definition}[Iloczyn kartezjański]
                Iloczyn kartezjański zbiorów A i B to zbiór, do którego należą wszystkie pary
                uporządkowane $ (a, b) $, gdzie $ a \in A $, $ b \in B $.

                \begin{equation*}
                    A \times B = \{ x: \exists a \in A, \exists b \in B, x = (a, b) \}
                \end{equation*}
                \begin{equation*}
                    \text{Piszemy  } A^2 = A \times A
                \end{equation*}
            \end{definition}

            Uogólnienie na trójki uporządkowane:
            \begin{equation*}
                (a, b, c) = ((a,b), c)
            \end{equation*}
            \begin{equation*}
                A_1 \times \ldots \times A_n = \{ (a_1, \ldots, a_n), a_i \in A_i 
                : \forall i = 1, \ldots, n \} = \times_{i=1}^{n} A_i
            \end{equation*}

            \textbf{Przykład.}
            \begin{align*}
                A \times (B \cap C) &= (A \times B) \cap (A \times C)\\
                & \subset \\
                & \supset
            \end{align*}

            \textbf{Dowód.}
            \begin{itemize}
                \item "$ \subset $" \\
                    Pokażemy, że $ A \times (B \cap C) = (A \times B) \cap (A \times C) $. \\
                    Niech $ (x, y) \in A \times (B \cap C) $. Wtedy $ x \in A \wedge y \in B \cap C $. \\
                    Zatem $ x \in A \wedge (y \in B \wedge y \in C) $. \\
                    Mamy $ (x \in A \wedge y \in B) \wedge (x \in A \wedge y \in C) $. \\
                    Stąd $ \underline{(x, y)} \in A \times B $ i $ \underline{(x, y)} \in A \times C $. \\
                    Zachodzi $ \underline{(x, y) \in (A \times B) \cap (A \times C)} $.

                \item "$ \supset $" \\
                    Pokażemy, że $ (A \times B) \cap (A \times C) \subset A \times (B \cap C) $. \\
                    Niech $ \underline{(x, y) \in (A \times B) \cap (A \times C)} $. \\
                    Stąd $ (x, y) \in A \times B \wedge (x, y) \in A \times C $. \\
                    Wtedy $ (x \in A \wedge y \in B) \wedge (x \in A \wedge y \in C) $. \\
                    Otrzymujemy $ x \in A \wedge (y \in B \wedge y \in C) $, tzn.
                    $ x \in A \wedge y \in B \cap C $. \\
                    Ostatecznie $ \underline{(x, y) \in A \times (A \cap C)} $

                \item Ponieważ $ \subset $ i $ \supset $ to
                    \begin{equation*}
                        A \times (B \cap C) \iff (A \times B) \cap (A \times C)
                    \end{equation*}
            \end{itemize}

            \pagebreak
            \textbf{Przykład.}
            \begin{align*}
                (A \setminus B) \times C) = (A \times C) \setminus (A \times C)\\
            \end{align*}

            \textbf{Dowód.}
            \begin{itemize}
                \item "$ \subset $"
                    \begin{align*}
                        \text{Niech  } (x, y) \in (A \setminus B) \times C &\iff 
                        ((x \in A \setminus B) \wedge (y \in C)) \\
                        &\iff ((x \in A \wedge x \notin B) \wedge (y \in C)) \\
                        &\iff ((x \in A \wedge \neg x \in B) \wedge (y \in C)).
                    \end{align*}

                \item "$ \supset $"
                    \begin{align*}
                        \text{Niech  } (x, y) \in (A \times C) \setminus (B \times C) &\iff \\
                        &\iff((x, y) \in A \times C) \wedge ((x, y) \notin B \times C) \\
                        &\iff(x \in A \wedge y \in C) \wedge (\neg (x, y) \in B \times C)  \\
                        &\iff(x \in A \wedge y \in C) \wedge \neg(x \in B \wedge y \in C)  \\
                        &\iff(x \in A \wedge y \in C) \wedge (\neg(x \in B) \wedge \neg(y \in C))
                    \end{align*}

                \item Oznaczamy
                    \begin{equation*}
                      \begin{cases}
                        &p: x \in A \\
                        &q : x \in B \\
                        &r : y \in C
                      \end{cases}
                    \end{equation*}
            \end{itemize}
        % subsection iloczyn_kartezjański (end)

        % section para_uporządkowana (end)
        \pagebreak
        \section{Relacje} % (fold)
        \label{sec:relacje}
        
        \subsection{Relacje} % (fold)
        \label{sub:relacje}
        
        % subsubsection relacje (end)
            \begin{definition}[Relacja]
                Relacją R w zbiorze $ X \times Y $ nazywamy dowolny podzbiór $ R \subset X \times Y $,
                R nazywamy relacją \underline{binarną}.
            \end{definition}

            \begin{definition}
                Niech R będzie relacją w zbiorze $ X $ tzn. $ R \subset X \times X $,
                
                \begin{enumerate}
                    \item R nazywamy \textbf{zwrotną} wtw.
                        \begin{equation*}
                             \forall x \in X : x R x
                         \end{equation*}
                    \item R nazywamy \textbf{przeciwzwrotną} wtw.
                        \begin{equation*}
                             \forall x \in X : \neg x R x
                         \end{equation*}
                    \item R nazywamy \textbf{symetryczną} wtw.
                        \begin{equation*}
                             \forall x, y \in X : ( x R y \implies y R x )
                         \end{equation*}
                    \item R nazywamy \textbf{przeciwsymetryczną} (asymetryczną) wtw.
                        \begin{equation*}
                             \forall x, y \in X : ( x R y \implies \neg(y R x) )
                         \end{equation*}
                    \item R nazywamy \textbf{słabo antysymetryczną} wtw.
                        \begin{equation*}
                             \forall x, y \in X : ( x R y \wedge y R x \implies x = y) )
                         \end{equation*}
                    \item R nazywamy \textbf{przechodnią} wtw.
                        \begin{equation*}
                             \forall x, y, z \in X : ( x R y \wedge y R z \implies x R z )
                         \end{equation*}
                    \item R nazywamy \textbf{spójną} wtw.
                        \begin{equation*}
                             \forall x, y \in X : ( \underbrace{x R y \vee y R x}_{\text{x i y są porównywalne}} \vee x = y )
                         \end{equation*}
                \end{enumerate}
            \end{definition}

            \pagebreak
            \begin{example}
                \begin{equation*}
                    \text{Niech   } R \subset \mathbb{R} \times \mathbb{R}
                \end{equation*}
                \begin{equation*}
                    x R y \iff x \leq |y|, \forall x, y \in \mathbb{R}.
                \end{equation*}
            \end{example}

            \begin{enumerate}
                \item R \textbf{zwrotna}?
                    \begin{equation*}
                        \forall x \in \mathbb{R}: x \leq |x|
                    \end{equation*}
                    \begin{center}
                        \textbf{TAK}
                    \end{center}

                \item R \textbf{przeciwzwrotna}?
                    \begin{equation*}
                        \forall x \in \mathbb{R}: \neg x \leq |x|
                    \end{equation*}
                    \begin{center}
                        \textbf{NIE}, np. $ x = 1 $
                    \end{center}

                \item R \textbf{symetryczna}?
                    \begin{equation*}
                        \forall x, y \in \mathbb{R}: ( x \leq |y| \implies y \leq |x| )
                    \end{equation*}
                    \begin{center}
                        \textbf{NIE}, np. $ x = 0, y = 1 $
                    \end{center}

                \item R \textbf{przeciwsymetryczna}?
                    \begin{equation*}
                        \forall x, y \in \mathbb{R}: ( x \leq |y| \implies \neg y \leq |x| )
                    \end{equation*}
                    \begin{center}
                        \textbf{NIE}, np. $ x = y = 0 $
                    \end{center}

                \item R \textbf{słabo antysymetrczyna}?
                    \begin{equation*}
                        \forall x, y \in \mathbb{R}: ( x \leq |y| \wedge y \leq |x| \implies x = y )
                    \end{equation*}
                    \begin{center}
                        \textbf{NIE}, np. $ x = 2, y = -2 $
                    \end{center}

                \item R \textbf{przechodnia}?
                    \begin{equation*}
                        \forall x, y, z \in \mathbb{R}: ( x \leq |y| \wedge y \leq |z| \implies x \leq |z| )
                    \end{equation*}
                    \begin{center}
                        \textbf{NIE}, np. $ x = 3, y = -5, z = 2 $
                    \end{center}

                \item R \textbf{spójna}?
                    \begin{equation*}
                        \forall x, y \in \mathbb{R}: ( x \leq |y| \vee y \leq |x| \vee x = y )
                    \end{equation*}
                    \begin{center}
                        \textbf{TAK}, ponieważ $ \forall x, y \in \mathbb{R} : ( x = y \vee x < y \vee y < x ) $\\
                        oraz $ \forall w \in \mathbb{R} : w \leq |w| $\\
                        zatem $ \forall x, y \in \mathbb{R}: ( x < |y| \vee y < |x| \vee x = y ) $ \\
                        stąd R spójna
                    \end{center}
            \end{enumerate}
            R jest zwrotna i spójna.

            \pagebreak
            \begin{example}
                \begin{equation*}
                    \text{Niech   } R \subset \mathbb{Z} \times \mathbb{Z}
                \end{equation*}
                \begin{equation*}
                    x R y \iff 3 \mid (x - y), \forall x, y \in \mathbb{Z}.
                \end{equation*}
            \end{example}

            \begin{enumerate}
                \item R \textbf{zwrotna}?
                    \begin{equation*}
                        \forall x \in \mathbb{Z}: 3 \mid (x - x)
                    \end{equation*}
                    \begin{center}
                        \textbf{TAK}
                    \end{center}

                \item R \textbf{przeciwzwrotna}?
                    \begin{equation*}
                        \forall x \in \mathbb{Z}: \neg (3 \mid (x - x))
                    \end{equation*}
                    \begin{center}
                        \textbf{NIE}, np. $ x = 0 $
                    \end{center}

                \item R \textbf{symetryczna}?
                    \begin{equation*}
                        \forall x, y \in \mathbb{Z}: (3 \mid (x - y) \implies 3 \mid (y - x) )
                    \end{equation*}
                    \begin{center}
                        \textbf{TAK}: Niech $ x R y \iff \exists k \in \mathbb{Z} : x - y = 3k $,
                                      wtedy $ \exists l = k \in \mathbb{Z} $ takie, że $ y - x = 3l $. \\
                                      Stąd $ y R x $.
                    \end{center}

                \item R \textbf{przeciwsymetryczna}?
                    \begin{equation*}
                        \forall x, y \in \mathbb{Z}: ( 3 \mid (x - y) \implies \neg (3 \mid (y - x)) )
                    \end{equation*}
                    \begin{center}
                        \textbf{NIE}, np. $ x = 6, y = 3 $
                    \end{center}

                \item R \textbf{słabo antysymetrczyna}?
                    \begin{equation*}
                        \forall x, y \in \mathbb{Z}: ( 3 \mid (x - y) \wedge 3 \mid (y - x) \implies x = y )
                    \end{equation*}
                    \begin{center}
                        \textbf{NIE}, np. $ x = 6, y = 3 $
                    \end{center}

                \pagebreak
                \item R \textbf{przechodnia}?
                    \begin{equation*}
                        \forall x, y, z \in \mathbb{Z}: ( 3 \mid (x - y) \wedge 3 \mid (y - z) \implies 3 \mid (x - z) )
                    \end{equation*}
                    \begin{center}
                        \textbf{TAK}: Niech $ 3 \mid (x - y) \wedge 3 \mid (y - z) $. \\
                                      Stąd $ x - y = 3k, k \in \mathbb{Z} $ \\
                                      $ y - z = 3l, l \in \mathbb{Z} $. \\
                                      Zatem $ x - y + y - z = 3k + 3l $ \\
                                      $ x - z = 3(k + l) $. \\
                                      Niech $ m = k + l, m \in \mathbb{Z} $, bo wtedy \\
                                      $ x - z = 3m $, tzn. $ 3 \mid (x - z) $.
                    \end{center}

                \item R \textbf{spójna}?
                    \begin{equation*}
                        \forall x, y \in \mathbb{Z}: ( 3 \mid (x - y) \vee 3 \mid (y - x) \vee x = y )
                    \end{equation*}
                    \begin{center}
                        \textbf{NIE}, np. $ x = 2, y = 1 $, bo $ ( 2 \neq 1 \wedge \neg(3 \mid 1) \wedge \neg(3(-1))) $
                    \end{center}
            \end{enumerate}
            R jest zwrotna, symetryczna oraz przechodnia.

            \pagebreak
            \begin{definition}[Relacja n-argumentowa, binarna, pusta, pełna]
                Niech $ X_1, \ldots, X_n $ będą zbiorami. Każdy podziór $ R \subset X_1 \times \ldots \times X_n $
                nazywamy relacją n-argumentową. \\
                Jeżeli $ n = 2 $, to $ R \subset X_1 \times X_2 $
                nazywamy relacją binarną. \\
                Jeżeli $ R = \varnothing $ to R nazywamy relacją pustą. \\
                Jeżeli $ R = X_1 \times \ldots \times X_n $, to R nazywamy relacją pełną.
            \end{definition}

            \begin{definition}[Dziedzina relacji]
                \textbf{Dziedziną} relacji $ R \subset X \times Y $ nazywamy
                \begin{equation*}
                    D(R) = \{ x \in X : \exists y \in Y : (x, y) \in R \} \subset X
                \end{equation*}
            \end{definition}

            \begin{definition}[Przeciwdziedzina relacji]
                \textbf{Przeciwdziedziną} relacji $ R \subset X \times Y $ nazywamy
                \begin{equation*}
                    D^\ast(R) = \{ y \in Y : \exists x \in X : (x, y) \in R \} \subset Y
                \end{equation*}
                
            \end{definition}

            \subsection{Relacja równoważności} % (fold)
            \label{sub:relacja_równoważności}

            \begin{definition}[Relacja równoważności]
                Relację $ R \subset X \times X $ nazywamy \underline{relacją równoważności} wtw.
                R jest \textbf{zwrotna, symetryczna i przechodnia}. Relacja $ R = \varnothing $
                jest relacją równoważności.
            \end{definition}

            \begin{definition}[Klasa abstrakcji]
                Niech $ R \subset X \times X $ relacja równoważności, $ X \neq \varnothing, x \in X $. \\
                \underline{Klasą abstrakcji }(klasą równoważności) elementu x (względem relacji R)\\
                nazywamy $ [x]_R = \{ y \in X : x R y \} \subset X $.
            \end{definition}

            \begin{definition}[Zbiór ilorazowy]
                \underline{Zbiorem ilorazowym} zbioru X przez relację R nazywamy zbiór wszystkich
                klas abstrakcji.
                \begin{equation*}
                    X / R = \{ [x]_R : x \in X \} \subset X
                \end{equation*}
            \end{definition}
            \hfill \\
            \pagebreak
            \begin{example}
                Niech $ X = \{ 1, 2, \ldots, 16 \}, R \subset X \times X, x R y 
                \iff 4 \mid (x^2 - y^2), \forall x, y \in X $.
            \end{example}
            Wyznaczyć zbiór ilorazowy $ X / R $. \\
            Czy R jest relacją równoważności?
                
            \begin{equation*}
              \begin{cases}
                \text{R zwrotna,} \forall x \in X : 4 \mid 10 \\
                \text{R symetryczna,} \forall x, y \in X : (4 \mid (x^2 - y^2) \implies 4 \mid (y^2 - x^2)) \\
                \text{R przechodnia,} \forall x, y, z \in X : (4 \mid (x^2 - y^2) \wedge 4 \mid (y^2 - z^2) \implies 4 \mid (x^2 - z^2))
              \end{cases}
            \end{equation*}
              $ \ldots \text{dowód} \ldots $\\\\
              Co to jest $ [x]_R $? $ x \in X $.

            \begin{equation*}
              \begin{aligned}
                  [x]_R &= \{ y \in X: x R y \} \\
                  &= \{ y \in X : 4 \mid (x^2 - y^2) \} \\
                  &= \{ y \in X: \exists k \in \mathbb{Z}: x^2 - y^2 = 4k \}
              \end{aligned}
            \end{equation*}

            \begin{equation*}
                4 \mid (x^2 - y^2) \iff x^2 - y^2 = 4k, k \in \mathbb{Z}
            \end{equation*}

            \begin{equation*}
              \begin{aligned}
                \text{np. }& [1]_R = \{ y \in X : y^2 = 1 + 4k, k \in \mathbb{Z} \} = \{ 1, 3, 5, \ldots \} \\
                &[2]_R = \{ y \in X : y^2 = 4 + 4k, k \in \mathbb{Z} \} = y^2 = 4(1 + k) = \{2, 4, \ldots \} \\
                &[3]_R = \{ y \in X : y^2 = 9 + 4k, k \in \mathbb{Z} \} = [1]_R \\
                &[4]_R = [2]_R \\
                &\vdots
              \end{aligned}
            \end{equation*}
            Dwie klasy abstrakcji. $ [1]_R = A_1, [2]_R = A_2 $. \\
            Zbiór ilorazowy $ X / R = \{ [x]_R : x \in X \} = \{ A_1, A_2 \} $
            \pagebreak

            \begin{lemat}
                Niech $ X \neq \varnothing, R \subset X \times X $ relacja równoważności. Wtedy
                \begin{equation*}
                    x R y \iff [x]_R = [y]_R, \forall x, y \in X
                \end{equation*}
            \end{lemat}

            \textbf{Dowód.}
            \begin{itemize}
                \item "$ \implies $"\\
                    Niech $ x, y \in X $, takie że $ x R y $.

                    \begin{itemize}
                        \item Pokażemy, że \framebox[1.1\width]{$ [x]_R \subset [y]_R $}. \\
                            Niech $ a \in [x]_R $. Wtedy \underline{$ a R x $}. Ponieważ równocześnie \underline{$ x R y $},
                            to wtedy \underline{$ a R y $} (z przechodniości R). \\
                            Zatem $ a R y $, więc $ a \in [y]_R $.
                            Stąd \framebox[1.1\width]{$ [x]_R \subset [y]_R $}.

                        \item Pokażemy, że \framebox[1.1\width]{$ [y]_R \subset [x]_R $}. \\
                            Niech $ b \in [y]_R $. Wtedy \underline{$ b R x $} (lub yRb, R symetryczne).
                            Ponieważ równocześnie \underline{$ x R y $ i $ y R b $},
                            więc \underline{$ x R b $} (lub bRx, z przechodniości R). \\
                            Stąd $ b \in [x]_R $.
                            Stąd \framebox[1.1\width]{$ [y]_R \subset [x]_R $}.

                    \end{itemize}

                    Wniosek: $ [x]_R = [y]_R $

                \item "$ \impliedby $"\\
                    Niech $ x, y \in X $, takie że $ [x]_R = [y]_R $.\\
                    Czy $ x R y $?\\

                    R zwrotna: $ x R x $. Zatem $ x \in [x]_R \ass [y]_R $. Więc $ x \in [y]_R $.
                    Stąd $ x R y $.
            \end{itemize}
            \pagebreak
            % subsection relacja_równoważności (end)
            
            \begin{definition}[Podział zbioru]
                Niech $ X \neq \varnothing $, P rodzina podzbioru zbioru $ X. (P \subset \mathcal{P}(x)) $\\
                Rodzinę P nazywamy \underline{podziałem zbioru X}, jeżeli

                \begin{itemize}
                    \item $ \forall A \in P, A \neq \varnothing $
                    \item $ \forall A, B \in P : (A \neq B \implies A \cap B \neq \varnothing) $
                    \item $ \bigcup P = X $
                \end{itemize}

                \textbf{To znaczy} P jest rodziną zbiorów nieparzystych, parami rozłącznych
                i jej sumą jest cały zbiór X.\\
                Zbiory rodziny P nazywamy \underline{blokami}.
            \end{definition}

            \textbf{Uwaga.} Jeżeli $ X \neq \varnothing $, to podział $ P = \{ \{ a \} : a \in X \} $
            jest "najdrobniejszy", a $ P = \{ X \} $ (jednoelementowy) zawiera tylko jeden blok.


            \begin{theorem}[Zasada abstakcji]
                Niech $ X \neq \varnothing $. Wtedy:
                \begin{enumerate}
                    \item Jeżeli $ R \subset X \times X $ jest \underline{relacją równoważności}
                    to wtedy zbiór ilorazowy $ X / R $ jest \underline{podzbiorem X}.
                    \item Jeżeli P jest podzbiorem zbioru X, to wtedy relacja:
                    \begin{equation*}
                        \begin{cases}
                            R \subset X \times X\\
                            x R y \iff \underbrace{\exists C \in P : (x \in C \wedge y \in C)}_
                            \text{"relacja pozostawania w tym samym zbiorze"}
                        \end{cases}
                    \end{equation*}
                jest relacją równoważności w X.
                \end{enumerate}
            \end{theorem}

            \begin{example}
                Czy suma i różnica dwóch relacji równoważności jest relacją równoważności?
            \end{example}

            \textbf{Nie.} $ X = \{ 1, 2, 3 \} $\\
            \begin{equation*}
                \text{są relacjami równoważności}\begin{cases}
                    R = \{ (1, 1), (2, 2), (3, 3), (1, 2), (2, 1) \}\\
                    S = \{ (1, 1), (2, 2), (3, 3), (1, 3), (3, 1) \}
                \end{cases}
            \end{equation*}
            Natomiast
            \begin{equation*}
                R \cup S = \{ (1, 1), (2, 2), (3, 3), (1, 2), (2, 1), (1, 3), (3, 1) \}
            \end{equation*}
            \begin{equation*}
                R \setminus S = \{ (1, 2), (2, 1) \}
            \end{equation*}
            ponieważ $ (2, 1), (1, 3) \in R \cup S $, ale $ (2, 3) \notin R \cup S $.
        % section relacje (end)
\end{document}