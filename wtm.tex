\documentclass[a5paper,8pt]{article}
\usepackage[utf8]{inputenc}
\usepackage{lmodern}
\usepackage[MeX]{polski}
\usepackage{amsmath}
\usepackage{amsthm}
\usepackage{amsfonts}
\usepackage{cases}
\usepackage{geometry}
\usepackage{enumitem}

\newgeometry{tmargin=2cm, bmargin=2cm, lmargin=1.5cm, rmargin=1.5cm}
\setlength{\parindent}{0cm}

\title{Wstęp do teorii mnogości}
\author{Stanisław Migórski}
\date{}
\frenchspacing

\begin{document}
    \maketitle

    \section{Program} % (fold)
    \label{sec:program}
        \begin{enumerate}
            \item Dowody i elementy logiki.
            \item Zbiory i działania na nich.
            \item Relacje równoważności.
            \item Funkcje.
            \item Własności funkcji.
            \item Zbiory równoliczne i nierównoliczne.
            \item Relacje porządku.
            \item Konstukcje liczbowe.
            \item Lemat Kuratowskiego-Zorna.
        \end{enumerate}
    % section program (end)

    \section{Literatura} % (fold)
    \label{sec:literatura}
        \begin{enumerate}
            \item K. Kustowski, A. Mostowski, \textit{Teoria mnogości}, PWN, 1994
            \item H. Rosiowa, \textit{Wstęp do matematyki}, PWN, 2004
            \item W. Marek, J. Onyszkiewicz, \textit{Elementy logiki i teorii mnogości w zadaniach}, PWN, 1996
        \end{enumerate}
    % section literatura (end)

    \section{Zasady oceniania} % (fold)
    \label{sec:zasady_oceniania}

        \begin{equation*}
          \textbf{WTM:  }\begin{cases}
            & \text{ćwiczenia 30h, \textbf{2} obecności bez usprawiedliwienia}.\\
            & \text{wykład 30h}.
          \end{cases}
        \end{equation*}

        \textbf{Ocenia końcowa: } \textbf{20\%} \underline{oceny} z ćwiczeń + \textbf{80\%} \underline{oceny} z egzaminu (I, II termin).\\

        \textbf{Pegaz: } zestawy zadań: A, B - obowiązkowe. Dowody do oceny "DDO".

    % section zasady_oceniania (end)

\end{document}